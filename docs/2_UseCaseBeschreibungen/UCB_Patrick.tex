\newlength\xlength
\settowidth\xlength{Alternativszenarien}
\addtolength\xlength{4\tabcolsep}

\newlength\ylength
\setlength\ylength\mylength
\addtolength\ylength{-\xlength}

\subsection*{Profil Individualisieren}
\label{tab:UCB_ProfilIndividualisieren}
\begin{tabularx}{\textwidth}{|>{\colorcelllight{}}l|X|X|X|X|}
	\hline
	\multicolumn{5}{|l|}{\colorcell{UseCase Beschreibung}}\\
	\hline
	Name&\multicolumn{4}{p{\ylength}|}{Profil individualisieren}\\
	\hline
	Kurzbeschreibung&\multicolumn{4}{p{\ylength}|}{Benutzer kann Gesichtspunkte seines Profils gestallten.}\\
	\hline
	Akteure&\multicolumn{4}{p{\ylength}|}{Benutzer}\\
	\hline
	Auslöser&\multicolumn{4}{p{\ylength}|}{Benutzer}\\
	\hline
	Eingehende Daten&\multicolumn{4}{p{\ylength}|}{Name, Geburtsdatum, Wohnort, Spielliste, Biographie}\\
	\hline
	Vorbedingungen&\multicolumn{4}{p{\ylength}|}{Keine}\\
	\hline
	Nachbedingungen&\multicolumn{4}{p{\ylength}|}{Aktualisiertes Nutzerprofil}\\
	\hline
	Essentielle Schritte&\multicolumn{4}{p{\ylength}|}{
	1. Benutzer authentifizieren.
	\newline
	2. Benutzer gibt zu erneurnde Profildaten ein.
	\newline
	3. Benutzerprofil aktualisieren.
	}\\
	\hline
	Alternativszenarien&\multicolumn{4}{p{\ylength}|}{
	Nutzer gibt fehlerhafte Daten ein.
	\newline
	1. Fehlermeldung anzeigen.
	\newline
	2. Nutzer zur erneuten Eingabe auffordern.
	}\\
	\hline
	Offene Punkte&\multicolumn{4}{p{\ylength}|}{}\\
	\hline
	Änderungshistorie&{\footnotesize\underline{\texttt{Wann}}}&{\footnotesize\underline{\texttt{Wer}}}&{\footnotesize\underline{\texttt{Neuer Status}}}&{\footnotesize\underline{\texttt{Was}}}\\
	&15.05.18&\hyperref[person:PatrickGruber]{Patrick Gruber}&Fertig&Erstellung\\
	\hline
	Sonstiges&\multicolumn{4}{p{\ylength}|}{}\\
	\hline
	Ersteller&\multicolumn{4}{p{\ylength}|}{\hyperref[person:PatrickGruber]{Patrick Gruber}}\\
	\hline
\end{tabularx}

\subsection*{Foreneintrag Schreiben}
\label{tab:UCB_ForeneintragSchreiben}
\begin{tabularx}{\textwidth}{|>{\colorcelllight{}}l|X|X|X|X|}
	\hline
	\multicolumn{5}{|l|}{\colorcell{UseCase Beschreibung}}\\
	\hline
	Name&\multicolumn{4}{p{\ylength}|}{Foreneintrag schreiben}\\
	\hline
	Kurzbeschreibung&\multicolumn{4}{p{\ylength}|}{Benutzer kann einen Eintrag im Forum verfassen und posten}\\
	\hline
	Akteure&\multicolumn{4}{p{\ylength}|}{Benutzer, Forenadmin}\\
	\hline
	Auslöser&\multicolumn{4}{p{\ylength}|}{Benutzer, Forenadmin}\\
	\hline
	Eingehende Daten&\multicolumn{4}{p{\ylength}|}{Text, den der Benutzer als Eintrag im Forum posten möchte}\\
	\hline
	Vorbedingungen&\multicolumn{4}{p{\ylength}|}{Benutzer authetifiziert}\\
	\hline
	Nachbedingungen&\multicolumn{4}{p{\ylength}|}{Foreneintrag erstellt und im Forum dargestellt}\\
	\hline
	Essentielle Schritte&\multicolumn{4}{p{\ylength}|}{
	1. Benutzer authetifizieren.
	\newline
	2. Benutzer verfasst Text.
	\newline
	3. Eintrag wird gepostet.
	}\\
	\hline
	Alternativszenarien&\multicolumn{4}{p{\ylength}|}{
	Benutzer erstellt Kopie eines anderen Beitrags.
	\newline
	1. Fehlermeldung ausgeben, dass Eintrag bereits existiert.
	\newline
	2. Benutzer Link zu diesem Eintrag anzeigen.
	}\\
	\hline
	Offene Punkte&\multicolumn{4}{p{\ylength}|}{}\\
	\hline
	Änderungshistorie&{\footnotesize\underline{\texttt{Wann}}}&{\footnotesize\underline{\texttt{Wer}}}&{\footnotesize\underline{\texttt{Neuer Status}}}&{\footnotesize\underline{\texttt{Was}}}\\
	&15.05.18&\hyperref[person:PatrickGruber]{Patrick Gruber}&Fertig&Erstellung\\
	\hline
	Sonstiges&\multicolumn{4}{p{\ylength}|}{}\\
	\hline
	Ersteller&\multicolumn{4}{p{\ylength}|}{\hyperref[person:PatrickGruber]{Patrick Gruber}}\\
	\hline
\end{tabularx}
